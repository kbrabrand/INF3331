\documentclass{article}

\usepackage{fancyvrb}
\usepackage[T1]{fontenc}
\usepackage[utf8]{inputenc}

\title{Obligatorisk innlevering 3, høsten 2014, INF3331}
\author{Kristoffer Brabrand <kristrek@student.matnat.uio.no>}
\date{\today}

\begin{document}
\maketitle

\section*{Innledende beskrivelse}
Oppgaven var grei nok å løse når det kom til prinsipper og bruk av python, men matematikken bød på noen utfordringer. Disse var primært knyttet til omregning mellom de ulike fargerommene, avrunding, grenseverdier mv.

De beskrevne oppgavene er løst, men jeg oppdaget for sent at denoise.c opererer på floats og konverterer til en unsigned char (0-255). Ettersom jeg opererer på heltall og runder av underveis har jeg noe større avvik i oppgaven. Dette er beskrevet under den aktuelle oppgaven.

Frontend for de ulike backendene er implementert i denoise.py i roten av oblig3-mappen, og er relativt enkel. Den gjør i korte trekk følgende;

\begin{enumerate}
  \item Parsing av kommandolinjeparametre.
  \item Validerer input.
  \item Henter riktig backend ved bruk av den lokale symboltabellen.
  \item Kaller \verb;denoise_file; på den valgte backenden;
  \begin{enumerate}
  	\item Laster fil inn i liste/numpy.ndarray.
  	\item Utfører denoising og/eller manipulasjon av bildet.
  	\item Skriver bearbeidet fil til målfil.
  \end{enumerate}
\end{enumerate}

Parametrene denoise.py tar vises enklest ved å kalle den med -h som parameter. Outputen som gis når man kaller \verb;denoise.py -h; er vist nedenfor.

%@exec python denoise.py -h

\section*{Oppgave 1: Denoise i Python og Numpy+Weave}

Python-implementasjonen er basert på algoritmen som fantes i denoise.c. Algoritmen gjør som den skal, men er (som forventet) ganske treg.

Numpy-weave-implementasjonen derimot er ganske mye raskere. Jeg har delt opp implementasjonen av C-koden i to biter og lagt disse ut i en separat python-fil for å gjøre \verb;numpy_weave.py; litt ryddigere. Koden finnes i src/denoise/weave\_c.py, og er delt i denoise\_c og support\_c, der den siste inneholder typedefs og funksjoner som brukes i denoising og manipulering.

I oppgaveteksten var det nevnt at man skulle bruke timeit for å sammenligne de ulike backendene. Det sto ikke eksplisitt hvor det skulle gjøres, men ettersom flere backends skjeldent kjøres samtidig og man derfor ikke enkelt vil kunne få en sammenligning ved å kjøre det gjennom frontend implementerte jeg en sammenligningstest i \verb;test_speed_test.py;. Testen sammenligner bare svart-hvit-denoising siden dette er den logikken som finnes i alle tre backendene.

\subsection*{Resultat fra kjøring}
%@exec python test_speed_test.py

Det eneste overraskende med resultatet er at min numpy-weave-implementasjon ser ut til å være raskere enn denoise\_c.

Det er egentlig ikke tilfelle, og skyldes at jeg i implementasjon av den rene C-backenden starter med å gjøre en import av bildet for å finne ut om det er et fargebilde (for å kunne gi feilmelding om at farger ikke støttes av den backenden).

\subsection*{Denoising-eksempler med ulike backends}

Nedenfor vises originalbildet, og videre vises dette bildet denoiset med en kappa på 0.1 og 10 iterasjoner med de ulike backendene; numpy-weave, pure-python og til slutt fra denoise.c.

\begin{figure}[!h]
\centering
\includegraphics[width=90mm]{disasterbefore}
\caption{Original image \label{original}}
\end{figure}

\pagebreak

\begin{figure}[!h]
\centering
%@exec
python denoise.py disasterbefore.jpg out.jpg --kappa=0.1 --iter=10
%@
\includegraphics[width=90mm]{nw-01-10}
\caption{Denoised with numpy-weave, kappa=0.1, iter=10 \label{nw-mono}}
\end{figure}

\pagebreak

\begin{figure}[!h]
\centering
%@exec
python denoise.py disasterbefore.jpg out.jpg --kappa=0.1 --iter=10 --backend=python
%@
\includegraphics[width=90mm]{p-01-10}
\caption{Denoised with python, kappa=0.1, iter=10 \label{p-mono}}
\end{figure}

\pagebreak

\begin{figure}[!h]
\centering
%@exec
python denoise.py disasterbefore.jpg out.jpg --kappa=0.1 --iter=10 --backend=c
%@
\includegraphics[width=90mm]{nw-01-10}
\caption{Denoised with denoise\_c, kappa=0.1, iter=10 \label{c-mono}}
\end{figure}

\pagebreak

\section*{Oppgave 1: Profilering}

Profileringen av de tre implementasjonene er gjort i \verb;test_profiling.py;. Jeg implementerte oppgaven før jeg så kodeeksempelet som ble lagt ut og falt derfor ned på en litt mindre elegant løsning med splitting og offsetting for å finne riktig utvalg fra outputen.

%@import
s = StringIO.StringIO();
ps = pstats.Stats(pr, stream=s).sort_stats('cumulative');
ps.print_stats();

print "\n\n==================================================";
print '{:=^50}'.format(" " + denoiser + " ");
print "==================================================";

print "\n".join(s.getvalue().splitlines()[4:8]);
%@

\subsection*{Resultat fra kjøring}
%@exec python test_profiling.py

Resultatet viser nokså tydelig hastighetsforskjellen mellom den rene python-implementasjonen og numpy-weave/denoise\_c – som er tilnærmet like raske. \verb;cProfile; er et fint verktøy for å profilere og tune python-kode, men er ikke spesielt hjelpsomt hvis målet er å optimalisere C-kode. Da er trolig \verb;gprof; et mer passende verktøy.

\subsection*{Kommentarer til implementasjonen}
Det er imidlertid verdt å nevne at svart-hvit-delen av koden, som er basert på denoise.c har lite rom for optimalisering ettersom den er såpass enkel og bare opererer på array-indekser, mens fargedelen med fordel kunne vært optimalisert.

Spesielt tenker jeg da på at den regner ut HSI-verdier fra RGB for omkringliggende punkter på nytt for hvert punkt. Den burde ha spart på, og gjenbrukt allokerte minneplasseringer for HSI og RGB-verdier og kunne nok med fordel også iterert over og konvertert alle piksler til HSI-verdier før manipuleringen begynte.

\section*{Oppgave 3: Kodeformatering}

Kodeformateringen i oppgaven håndteres med verbatim-blokker i det ferdige latex-dokumentet. Ettersom dette er noe som brukes av nesten alle modulene har jeg valgt å skille ut denne logikken i egne metoder; verbatim\_exec og verbatim\_code i verbatim-modulen.

Ved å kalle disse uten pretty-parametert (eller med parameteret satt til False) omsluttes stringen som sendes inn i metodenen som første parameter (navngitt hhv. code og result) av instruksjoner som gir en enkel vebatim-blokk.

Ved å sende med True for pretty-parametert gis det en litt stiligere verbatim-blokk.

Verbatim-metodene kan brukes slik;

%@import
# Fancy verbatim exec
print verbatim_exec('echo 2', pretty=True);

# Plain verbatim code
print verbatim_code('print 2');
%@

Kodeformatering gjøres som del av de ulike leddene i prosesseringen som gjøres når prepro.py kjøres på en latex-fil. Implementasjonen finnes i src/verbatim.py.

\section*{Oppgave 4: Vanlig innliming av kode}

I denne oppgaven bruker jeg i likhet med de fleste andre verbatim-modulen for formatering. Denne oppgaven ble med dette redusert til å gjøre en multiline-regex for å finne kodeblokker og deretter kalle på \verb;verbatim.verbatim_code; eller \verb;verbatim.verbatim_exec; avhengig av om det var en \verb;%import;- eller \verb;\%@exec;-instruksjon.

Jeg bruker dette regulære uttrykket for å finne kode som er limt inn.

%@import
(%@(import|exec)\n((.*\n)+?)%@)
%@

Når uttrykket matches mot teksten gir det en gruppe med nøkkelordet (import eller exec) i, og jeg bruker dette til å finne ut om jeg skal pakke linjene mellom start og sluttaggen i code- eller exec-verbatim.

Formatering av innlimt kode er et av leddene i prosesseringen som gjøres når prepro.py kjøres på en latex-fil. Implementasjonen finnes i src/inline\_blocks.py.

\section*{Oppgave 5: Frontend}

Output fra frontend er vist allerede, så jeg gjentar ikke det, men et par andre ting er verdt å nevne.

\subsection*{Timeit i frontend}
Det står nevnt i opgaveteksten at det skal være mulig å skru \verb;timeit;-modulen av/på i frontend. Med en tanke om fordeling av ansvar (separation of concerns) i bakhodet mener jeg at det tilfører lite verdi å ha timeit-funksjonalitet i frontenden ettersom man skjeldent bruker mer enn én backend samtidig likevel.

Jeg valgte derfor, som tidligere beskrevet, å implementere hastighetssammenlikning i en separat fil; \verb;test_speed_test.py; og output fra denne er vist tidligere.

\subsection*{eps-parameter til frontend}
Jeg forsto rett og slett ikke hvorfor denne skulle være et parameter til frontend og valgte til slutt å ta den bort. Den er tatt med som første og eneste parameter til sammenlikningstesten som genererer og sammenlikner verdier i bilder (\verb;test_file_comparison.py;).

\section*{Oppgave 7: Kompilering av preprosessert fil}

Kompilatoren er så begrenset i omfang at jeg valgte å implementere den som én enkelt fil. Kompilatoren starter en subprosess av \verb;pdflatex; ved hjelp av \verb;subprocess.Popen;

Som standard kjøres \verb;pdflatex; i nonstopmode ved at interaction-flagget settes til nonstopmode. Dette kan deaktiveres ved å sette interactive-flagget når compile.py-scriptet kalles.
Output fra compile.py går – som i \verb;pdflatex; – som standard til den samme mappen som latex-filen som prosesseres ligger i. Dette kan overstyres ved å kalle compile.py med et destination-parameter.

Alle parametre til compile.py vises enkelt ved å kalle \verb;python compile.py -h;. Dette gir følende output;

%@exec python compile.py -h

\end{document}