\documentclass{article}

\usepackage{fancyvrb}
\usepackage[T1]{fontenc}
\usepackage[utf8]{inputenc}

\title{Obligatorisk innlevering 1 høsten 2014, INF3331}
\author{Kristoffer Brabrand <kristrek@student.matnat.uio.no>}
\date{\today}

\begin{document}
\maketitle

\section*{Oppgave 1.1}

Oppgaven er løst ved bruk av \verb;find;, \verb;xargs;, \verb;du; og \verb;sort;.

Først brukes \verb;find; til å lete etter filer med modifisert timestamp i løpet av  det antall dager spesifisert når skriptet kalles.

Resultatet pipes til \verb;xargs; som kaller på \verb;du;, som igjen returnerer størrelse (i kilobyte) og navn på hver fil.

Til slutt pipes resultatet videre til \verb;sort; som sorterer stigende.

\subsection*{Kjøring}
Når man står i roten av oppgavemappen kan skriptet kjøres med kommandoen under. Skriptet krever et heltall for dag-parameteret.
\begin{Verbatim}[fontsize=\small, frame=single]
$ ./bash/list_new_files path days
\end{Verbatim}

Eksempelresultat:
\begin{Verbatim}[fontsize=\small, frame=single]
 36K	file_tree/Kq0Wv/MH/zWG/Exw0zNwi
 36K	file_tree/Pkvye/htZiVgRE
 40K	file_tree/Kq0Wv/MH/zWG/LwgfcJ8
 48K	file_tree/Kq0Wv/5RYWI5kQ
 84K	file_tree/Kq0Wv/MH/gBwNRP
104K	file_tree/Kq0Wv/MH/7GvTL2y
176K	file_tree/Kq0Wv/MH/zWG/1zeD9ON
\end{Verbatim}

\section*{Oppgave 1.2}

Oppgaven er løst ved bruk av \verb;find;, \verb;xargs;, \verb;grep;.

Først brukes \verb;find; til å finne alle filer.

Resultatet pipes til \verb;xargs; som kaller på \verb;grep;, som leter etter et ord og returnerer linjenummeret og linjen der søkeordet ble funnet.

\subsection*{Kjøring}
Når man står i roten av oppgavemappen kan skriptet kjøres med kommandoen under.
\begin{Verbatim}[fontsize=\small, frame=single]
$ ./bash/list_new_files path word
\end{Verbatim}

Eksempelresultat:
\begin{Verbatim}[fontsize=\small, frame=single]
file_tree/_CVcim:541:V_3fxWfSRqJrarezHd
file_tree/Kq0Wv/5RYWI5kQ:1384:3jz26eTjD1CBiZ7kI5iU5FareaF
file_tree/Kq0Wv/5RYWI5kQ:1957:1i9icbd2qzTcgLB50LCFRZUFKfUarelQ
file_tree/Kq0Wv/MH/_Oj2c0QA:22304:csCA8TAfarezymEYdYtmGL_eB
file_tree/Kq0Wv/MH/Z9kP8NB:7194:KLrM0iDmhareiDWreks
file_tree/Kq0Wv/MH/Z9kP8NB:7992:9MifsLNareBB1gfBjADQWTcVUNT
[...]
\end{Verbatim}

\section*{Oppgave 1.3}

Oppgaven er løst ved bruk av \verb;find;. Alle filer med en størrelse over det antall kilobyte som er gitt skriptet som paramter returneres og \verb;find;s exec-parameter brukes for å returnere filnavnet på alle disse filene. I tillegg brukes delete-parameteret, sletter alle filer som passet til de andre kriteriene til find-kommandoen.

\subsection*{Kjøring}
Når man står i roten av oppgavemappen kan skriptet kjøres med kommandoen under. Skriptet krever et heltall for dag-parameteret.
\begin{Verbatim}[fontsize=\small, frame=single]
$ ./bash/sized_delete.sh path size
\end{Verbatim}

Eksempelresultat ved treff:
\begin{Verbatim}[fontsize=\small, frame=single]
Deleting...
file_tree/Kq0Wv/MH/_Oj2c0QA
file_tree/Kq0Wv/MH/XhdhBbk
file_tree/Kq0Wv/MH/Z9kP8NB
file_tree/Kq0Wv/MH/zWG/8puxfjS
file_tree/Kq0Wv/MH/zWG/Qww53eF
file_tree/LGPbdlRW
\end{Verbatim}
Eksempelresultat dersom skriptet ikke finner filer med størrelse over det angitte antall kilobytes:
\begin{Verbatim}[fontsize=\small, frame=single]
No files of size larger than 350 kilobytes found
\end{Verbatim}

\section*{Oppgave 1.4}

Oppgaven er løst ved bruk av \verb;sort;.

Sort tar en fil, sorterer linjene i stigende rekkefølge og returnerer linjene i sortert rekkefølge. Ved bruk av \textbf{-o}-parameteret skrives output fra sorteringen til fil i stedet for stdout.

\subsection*{Kjøring}
Når man står i roten av oppgavemappen kan skriptet kjøres med kommandoen under. Skriptet krever et heltall for dag-parameteret.
\begin{Verbatim}[fontsize=\small, frame=single]
$ ./bash/sort_file.sh source destionation
\end{Verbatim}

Eksempelresultat:
\begin{Verbatim}[fontsize=\small, frame=single]
$ cat unsorted
orange
pear
apple
grape
pineapple

$ ./bash/sort_file.sh unsorted sorted && cat sorted
apple
grape
orange
pear
pineapple
\end{Verbatim}

\section*{Oppgave 2}

Ved løsingen av oppgaven er det tatt utgangspunkt i malen som ble lagt ut på Github.

Det er brukt rekursjon for både generering og populering av mappetreet. Hjelp til bruk av filtre-generatoren gis ved å kalle skriptet uten parametre.

For å gjøre det enklere å se hva som er filer og hva som er mapper har jeg lagt til .file som filendelse på alle filene.

\subsection*{Kjøring}
Eksempelresultat ved kall uten parametre:
\begin{Verbatim}[fontsize=\small, frame=single]
$ ./python/generate_filetree.py
Not enough arguments included.
usage: ./python/generate_filetree.py target dirs files [size rec_depth
start end seed verbose]
\end{Verbatim}

Eksempelresultat ved kall med påkrevde parametre:
\begin{Verbatim}[fontsize=\small, frame=single]
$ ./python/generate_filetree.py testdir 2 2 && tree testdir
testdir
|-- XkVsYva
| |-- 1JBat2cj
| | |-- 4a6RLqt.file
| | |-- 7bdr.file
| |-- 3VAmq3IByiBW7Ov.file
| |-- BPi9ZWLjVoD
|     |-- K1BmIhzAZ4GiN.file
|     |-- rH.file
|-- bZK
    |-- QUH
    | |-- Kzqk3sJ5.file
    |-- Zrr2Kzh4P6jwSxG.file
    |-- iZ
        |-- i.file

6 directories, 8 files
\end{Verbatim}
\end{document}
